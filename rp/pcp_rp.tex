\begin{comment}
In this subsection we:
- describe PCP's protocol by AM
- describe certain assumptions on cost
- we prove that it is a Sequential Rational Proofs
\end{comment}

We now describe a rational proof appeared in \cite{am1} and prove that is sequentially composable.
The protocol assumes the existence of a trusted memory storage to which both 
Prover and Verifier have access, to realize the so-called "PCP" (Probabilistically Checkable Proof) model. In this model, the Prover 
writes a very long proof of correctness, that the verifier checks only in a few randomly selected position. The trusted memory is needed to 
make sure that the prover is "committed" to the proof before the verifier starts querying it. 
% say that it is for logical circuits
% say about the abuse of notation in the theorem

The following protocol for proofs on a binary logical circuit $\cal C$ 
appeared  in  \cite{am1}. The Prover writes all the (alleged) values $\alpha_w$ for every wire in $w \in {\cal C}$, 
on the trusted memory location. The Verifier 
samples a single randome gate value to check its correctness and determines the reward accordingly:
\begin{enumerate}
\item The Prover writes the vector $\{ \alpha_w\}_{w \in {\cal C}}$
\item The Verifier samples a random gate $g \in \cal C$.
\begin{itemize}
\item  The Verifier reads $\alpha_{g_{out}}, \alpha_{g_L}, \alpha_{g_R}$, with $g_{out}, g_L, g_R$ being respectively the output, left and right 
input wires of $g$; the verifier checks that $\alpha_{g_{out}} = g(\alpha_{g_L}, 
\alpha_{g_R})$;
\item If $g$ in an input gate the Verifier also checks that  $\alpha_{g_L}, \alpha_{g_R}$ correspond to the correct input values;
\end{itemize}
The Verifier pays $R$ if both checks are satisfied, otherwise it pays $0$.
\end{enumerate}

\XXX \textbf{Next Theorem Should Cite Azar-Micali!} %[\cite{am1}]
\begin{theorem}
The protocol above is a rational proof for any boolean function in $
P^{||NP}$, the class of all languages decidable by a polynomial time machine 
that can make non-adaptive queries to $NP$.
\end{theorem}

We will now show  a cost model where the rational proof above is sequentially 
composable.
We will assume that the cost for any prover is given by the number of gates he 
writes. Thus, for any input 
$x$, the costs for honest and dishonest provers are 
respectively $C(x) = S$, where $S = |\cal C|$, and $\tilde{C}(x) = \tilde{s}$ 
where $\tilde s$ is the number of gates written by the dishonest prover.
Observe that in this model a dishonest prover may not write all the $S$ 
gates, and that not all of the $\tilde{s}$ gates have to be correct. Let $\sigma \leq \tilde{s}$ the number of correct gates written by $\disP$. Then 

\begin{theorem}
In the cost model above the PCP protocol protocol in \cite{am1} is sequentially 
composable.
\end{theorem}
\begin{proof}
Observe that the probability $\pDisR$ that $\disP\neq P$ earns $R$ is such that
$$ \pDisR = \frac{\sigma}{S} \leq \frac{\tilde s}{S} = \frac{\tilde C}{C} $$
Applying Corollary \ref{cor:prob} completes the proof.
\end{proof}

The above cost model, basically says that the cost of writing down a gate dominates everything else, in particular the cost of {\em computing} that gate. 
In other cost models a proof of sequential composition may not 
be as straightforward. Assume, for example, that the honest prover pays $\$1$ to
compute the value of a single gate while writing down that gate is "free". Now $\pDisR$ is still equal to $\frac{\sigma}{S}$ but to 
prove that this is smaller than $\frac{\tilde C}{C}$ we need some additional assumption that limits the ability for $\disP$ to "guess" the 
right value of a gate without computing it (which we will discuss in the next Section). 


